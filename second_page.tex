\noindent\uline{\hfillХарківський національний університет радіоелектроніки\hfill} \\
Факультет \uline{комп’ютерних наук\hfill} \\
Кафедра \uline{програмної інженерії\hfill} \\
Напрям підготовки \uline{Програмна інженерія\hfill} \\
Курс \uline{\makebox[4em]{3}} Семестр \uline{\makebox[4em]{5}\hfill} \\
\vspace*{\baselineskip}
Навчальна дисципліна \uline{Аналіз та рефакторінг коду програмного забезпечення}
\vspace*{\baselineskip}
\begin{center}
  ЗАВДАННЯ \\
  НА КУРСОВУ РОБОТУ СТУДЕНТОВІ
\end{center}
\noindent\uline{\hfill Крилову Кирилу Юрійовичу \hfill} \\
\begin{enumerate}
  \item Тема роботи: \uline{\textit{«Програмна система для колективного прослуховування музики»}\hfill}
  \item Термін узгодження завдання курсової роботи «\uline{\hspace{1em}}»\uline{\hfill}2017 р.
  \item Термін здачі студентом закінченої роботи  «\uline{\textit{27}}»\uline{\hfill\textit{грудня}\hfill}2017р.
  \item Вихідні дані до проекту (роботи):
    \textit{\uline{
        Використовувати мови розробки Elixir та Javascript, середи розробки Neovim та Android Studio, PostgreSQL.\hfill
    }}
    \mbox{}
  \item Зміст пояснювальної̈ записки (перелік питань, що належить розробити) \\
    \textit{\uline{
        вступ, аналіз предметної області, постановка задачі, моделювання \hfill\mbox{}\\
        програмного продукту, проектування бази даних, опис інтерфейсу та функціоналу, висновки, перелік посилань.
        \hfill
    }}
    \mbox{}
  \item Перелік графічного матеріалу (з точним зазначенням обов’язкових креслень) \\
    \textit{\uline{
        діаграма класів, діаграма прецедентів, діаграма послідовностей, діаграма станів.
        \hfill
    }}
\end{enumerate}

\newpage
