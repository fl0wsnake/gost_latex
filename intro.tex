\begin{center}\chapter*{ВСТУП}\end{center}
\addcontentsline{toc}{chapter}{ВСТУП}

    Протягом майже усієї історії музика була невід’ємною частиною культури людей. Музика – могутній засіб всебічного розвитку людини, формування її духовного світу. Вона розширює наш кругозір, знайомить з різноманітними явищами, збагачує почуттями, викликає радісні переживання, сприяє вихованню правильного ставлення до навколишнього світу. На даний момент існує дуже багато способів знаходження музики, яка нам сподобається. Ми можемо почути музику на різних музичних телепередачах, на радіо, у кіно, в опері, на вулиці тощо. Однак з розвитком технологічного прогресу, таких засобів становиться ще більше.

    Найчастіше, включаючи телевізор, якщо в нас є чітко поставлена ціль щодо прослуховування музики, ми включаємо музичні канали. Це дуже зручно, тому що ми не можемо вибирати жанр музики, яку ми хочемо послухати. Також у нас не має змоги переключити пісню, якщо вона нам не подобається. Усі плейлисти складаються з пісень (треків), які створюють робітники музичних каналів, а вони зі свого боку складають плейлисти, засновуючись на популярних зараз треках. Врахувавши все зазначене вище, можна зробити висновок, що такий варіант прослуховування музики не є кращим на теперішній момент.

    Радіо не залишає перших місць серед засобів розважання і в наші дні. Якщо ви введете в будь-який пошуковик в Інтернеті «Слухати радіо онлайн», ви побачите безліч посилань. Це означає, що люди мають безперервний доступ до найсвіжіших новин у музиці на різноманітним смаки.

    На даний момент існують мільйони радіостанцій, які працюють цілодобово. Ці радіостанції розважають людей короткими випусками новин, проводять різноманітні обговорення, які допомагають самотнім людям уникати нестачі спілкування. Поширеність спортивних і музичних передач - це основні теми, які ми можемо почути на радіостанціях.

    В Інтернет-радіостанціях є більше функціоналу, порівняно з телебаченням. Ви можете вибирати різні радіостанції, виходячи зі своїх музичних вподобань.

    На даний момент ми можемо прослуховувати музику з різних сервісів. Наприклад, сервіс Spotify, Google Play Music, Pandora. Усі ці сервіси використовують сучасні технології для більш комфортного прослуховування або знаходження музики. Впровадження сучасних методів розробки дає більшу варіативність в використовуванні типових баз даних з музикою.

    Однак в наш час кожен легко може мати велику кількість улюблених музичних жанрів та виконавців завдяки багатьом стрімінговим сервісам (Інтернет-радіо) та загальній доступності інформації. Отже, під час групових подій, важливо мати засіб завжди слухати пісні, цікаві більшості людей, витрачаючи на сперечання якомога менше часу.

    Багато людей використовують плеєри для прослуховування музики на вулиці в транспорті тощо. Також сьогодні дуже великий попит мають smart-телефони або смартфони. Смартфони являються невід’ємною частиною нашого життя. За допомогою багатого функціоналу смартфонів ми можемо виконувати такі операції як веб-серфінг, та прослуховування музики поза зоною дому. Сьогодні кожна компанія або фірма, яка йде в ногу з часом, розуміє, що створити хороший сайт або надрукувати якісну поліграфію - недостатньо. У час, коли більшість людей мають під рукою свій портативний мікросвіт в смартфоні, єдине рішення для тих, хто хоче бути серед лідерів - створити якісні та зручні мобільні додатки.

    Розробка мобільних додатків – складний і багатоетапний процес, тому, починаючи його, слід визначити потреби своїх користувачів і клієнтів, функції, цей додаток повинен виконувати, а також визначитися з його типом і мобільною платформою.
