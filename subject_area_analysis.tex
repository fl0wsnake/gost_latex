\begin{center}\chapter{АНАЛІЗ ПРЕДМЕТНОЇ ОБЛАСТІ}\end{center}

На даний момент існує небагато систем, здатних забезпечити подібний функціонал:
\begin{enumerate}[label=\alph*)]
  \item
    Google Play Music – Інтернет-сервіс потокової музики, який підтримує потокову музику у веб-браузері ПК \imgref{google-play-music}.

    \img{google-play-music}{Інтерфейс Play Music для веб-браузера}

    Також даний сервіс дозволяє користувачам використовувати додаток для телефонів і планшетів з операційною системою Android чи будь-який інший девайс, який використовує Adobe Flash платформу.

    Музичний сервіс дозволяє користувачам зберігати свою музичну бібліотеку на комп’ютерах або плеєрах, а також на серверах Google, щоб слухати її з будь-якого кутка світу.
  \item
    Pandora – служба потокового відтворення музики в Інтернеті, засноване на рекомендаційної системі «Music Genome Project». Pandora, як і вищезазначений Google Play Music має додаток для телефонів і планшетів та веб-сайт для ПК \imgref{pandora}.

    \img{pandora}{Інтерфейс веб-сайту Pandora}

    Користувач медіа-програвача Pandora, створеного на основі платформи OpenLaszlo, вибирає музичного виконавця, після чого система шукає схожі композиції, використовуючи близько 400 музичних характеристик (наприклад, синкопа, тональність, гармонія і т. д.). Використовуючи функції «подобається» чи «не подобається», слухач часто може налаштувати радіостанцію на свій смак. У базі даних системи понад мільйон композицій і більше ста тисяч виконавців. Зареєстрований користувач може створити в своєму профілі до 100 різних «радіостанцій», транслюють музику в тих чи інших жанрах.

  \item
    Найближчим аналогом даного проекту є функція  Spotify з назвою «Collaborative Playlist» \imgref{spotify}.

    \img{spotify}{Інтерфейс Spotify Collaborative Playlist}

    Дана функція переносить трек до спільного списку пісень, який можуть прослуховувати та редагувати декілька користувачів. Тобто видаляти треки або переставляти їх у різному порядку. За допомогою цієї функції ви можете надсилати цікаві треки вашим друзям та прослуховувати їх разом. Але для даної функції не вистачає деяких функцій, які будуть розглянуті далі.

    Як і у кожного програмного продукту, в даній системі існують певні бізнес-ризики. Розроблюваний продукт може зустрітись з такими ризиками:

    \begin{enumerate}[label=\arabic*)]
      \item зміна курсів валют, процентних ставок, ставок і умов кредитів;
      \item бажання користувачів користуватись більш складними типами систем;
      \item витрати на розробку програмного забезпечення та технічних засобів.
    \end{enumerate}

    Однак основне призначення продукту – полегшити процес узгодження музики яка буде грати на будь-якій соціальній події. Розроблювана система буде надавати користувачам можливість створювати відкриті або закриті групи, де запрошені будуть мати безпосередній вплив на майбутні пісні через різноманітні голосування. Система буде мати інтерес серед усіх кругів спілкування, де хтось має принаймні деяку схильність до музики.

\end{enumerate}
\newpage
